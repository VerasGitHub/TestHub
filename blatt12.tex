\documentclass{uebungsblatt}
\setcounter{aufgabe}{3} 
\usepackage{amsmath,amssymb}
%makros
\newcommand{\R}{{\mathbb R}}
\newcommand{\Rn}{{\mathbb R^n}}  
\newenvironment{eqv*}{\begin{equation*}\hfill} {\hfill \end{equation*}} 
\newenvironment{eqv}{\begin{equation}\hfill} {\hfill \end{equation}}
\begin{document}
\section*{Hausaufgaben}
\begin{aufgabe*}{H33: Schwache Formulierung einer partiellen Differentialgleichung}
Es sei $\Omega := B(0, \frac 1 2) \subset \R^2$. Gesucht ist ein $u \in V:=H^1_0(\Omega)$, sodass
$$
\int_{\Omega} (D_1 u + x_1^2 D_2 u)D_1 v + (x_1^2 D_1 u + D_2 u) D_2 v = \int_{\Omega} \cos(x_1 x_2) v \quad \forall v \in V
$$
erf�llt ist. Untersuchen Sie die Aufgabe auf Existenz und Eindeutigkeit.

\end{aufgabe*}
\begin{aufgabe*}{H34: Die biharmonische Gleichung}
Es sei $\Omega \subset \Rn$ offen und beschr�nkt. 
\begin{teilaufgaben}{a)}
\teilaufgabe Es sei $H^k_0(\Omega):= \overline{C^{\infty}_0(\Omega)}^{\|\cdot\|_{H^k(\Omega)}}$. Zeigen Sie, dass
$$
\|u \|_{H^k_0(\Omega)} := \left(\sum_{|\alpha| = k}  \| D^{\alpha} u \|^2_{L^2(\Omega)} \right)^{\frac 1 2}
$$
eine zur $H^k$-Norm �quivalente Norm auf $H^k_0(\Omega)$ definiert.
\teilaufgabe Zeigen Sie, dass f�r alle $u \in C^{\infty}_0(\Omega)$ die folgende Gleichung gilt
$$
 \int_{\Omega} \! (\Delta u)^2 = \int_{\Omega}  \! |\nabla^2 u|^2.
 $$
\teilaufgabe Wir nennen $u \in H^2_0(\Omega)$ die schwache L�sung der \emph{biharmonischen Gleichung}
$$
\Delta^2 u = f \quad \text{in }\Omega, \quad u= \frac {\partial}{\partial \nu} u = 0 \text{ auf }\partial \Omega,
$$
wenn 
$$
\int_{\Omega} \! \Delta u \Delta v = \int_{\Omega} fv \quad \forall v \in H^2_0(\Omega).
$$
Zeigen Sie, dass f�r jedes $f \in L^2(\Omega)$ genau eine schwache L�sung der biharmonischen Gleichung existiert. 
Dabei wird die in \textbf{b)} bewiesene Gleichung n�tzlich sein.
\end{teilaufgaben}
\end{aufgabe*}
%%
\begin{aufgabe*}{H35: Satz von Riesz im reellen Hilbertraum}
Es sei $\left(H, ( \cdot, \cdot )_H\right)$ ein Hilbertraum �ber $\R$ und $f \in H^{'}$, d.h., $f: H \rightarrow \R$ ist linear und stetig.
Zeigen Sie: Dann existiert genau ein $u_f\in H$, sodass
\begin{eqv}
(u_f,v)_H = \langle f, v\rangle_{H^{'}\! \!, H}\quad \forall v \in H \label{Riesz}
\end{eqv}
gilt.

% Dies ist der Riesz'sche Darstellungssatz.
%
Anleitung: Betrachten Sie dass Minimierungsproblem 
$$
\min_{v\in H} F(v) := \frac 1 2 \|v\|_H^2 - \langle f,v\rangle_{H^{'}\! \!, H}
$$
und eine Minimalfolge, d.h. $v_n \in H$ mit $F(v_n) \rightarrow \inf_{v\in H} F(v) $. Beweisen Sie zun�chst, dass
$$
F\left(\frac{v_m+v_n}{2} \right) + \frac 1 2 \left \| \frac{v_m - v_n}{2} \right\|^2_H = \frac 1 2 ( F (v_m) + F(v_n))
$$
gilt. Was bedeutet das f�r die Minimalfolge? Danach zeigen Sie die Existenz und Eindeutigkeit des Minimierungsproblems. Schlie�en Sie dann die Gleichung \eqref{Riesz} aus der notwendigen Bedingung erster Ordnung.
\end{aufgabe*}
\vfill
\centering \textbf{Abgabe:} 5. Februar 2015 
\end{document}